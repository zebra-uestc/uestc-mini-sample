\chapter{绪论}

\section{研究工作的背景与意义}

计算电磁学方法\Cite{white2012hadoop,ghemawat2003google,ross2000pvfs}从
时、频域角度划分可以分为频域方法与时域方法两大类。频域方法的研究开展较
早,目前应用广泛的包括:矩量法(MOM)\Cite{banker2011mongodb}及其快速
算法多层快速多极子(MLFMA)\Cite{beck2002junit,beck2004junit}方法、有
限元(FEM)\Cite{hunt2010zookeeper}方法、自适应积分(AIM)%
\Cite{shvachko2010hadoop}方法等,这些方法是目前计算电磁学商用软
件\footnote{脚注,格式待调整。}(例如:FEKO、Ansys等)的核心算法。由文献%
\cite{banker2011mongodb,beck2004junit,hunt2010zookeeper,shvachko2010hadoop}
可知\ldots

\section{时域积分方程方法的国内外研究历史与现状}

时域积分方程方法的研究始于上世纪60年代,C.L.Bennet等学者针对导体目标的
瞬态电磁散射问题提出了求解时域积分方程的时间步进(marching-on in-time,
  MOT)算法。

\ldots

\section{本文的主要贡献与创新}

本论文以时域积分方程时间步进算法的数值实现技术、后时稳定性问题以及两层
平面波加速算法为重点研究内容,主要创新点与贡献如下:

\ldots

\section{本论文的结构安排}

本文的章节结构安排如下:

\ldots
